% normalization
\subsection{Data normalization}
I define my normalization function $f_{norm}:\mathbb{R}^{22}\to \mathbb{R}^{22}$ by:
\begin{align}
f_{norm}(x) = \left(f^1_{norm}(x_1), ...,f^{22}_{norm}(x_{22})\right) 
\end{align}
where
\begin{align}
f^i_{norm}(x_i) = \frac{x_i - \mu_i}{\sigma_i}
\end{align}
$\mu_i$ and $\sigma_i$ are here the empirical mean and empirical standard deviation.

I did do the computation with \texttt{sklearn} by using the \texttt{StandardScaler} from preprocessing in \texttt{sklearn}
to normalize the data.

Here we have the table of the mean and standard deviation, before and after the normalization of the training data:
\begin{center}
\begin{tabular}{rrrrr}
\toprule
 Feature &  before mean &  normalized mean &  before std. deviation &  normalized std. deviation \\
\midrule
       1 &     155.9604 &              0.0 &                44.3036 &                        1.0 \\
       2 &     204.8212 &              0.0 &                98.1520 &                        1.0 \\
       3 &     115.0586 &              0.0 &                45.7556 &                        1.0 \\
       4 &       0.0060 &              0.0 &                 0.0040 &                        1.0 \\
       5 &       0.0000 &              0.0 &                 0.0000 &                        1.0 \\
       6 &       0.0032 &              0.0 &                 0.0024 &                        1.0 \\
       7 &       0.0033 &              0.0 &                 0.0023 &                        1.0 \\
       8 &       0.0096 &              0.0 &                 0.0071 &                        1.0 \\
       9 &       0.0277 &              0.0 &                 0.0159 &                        1.0 \\
      10 &       0.2624 &              0.0 &                 0.1627 &                        1.0 \\
      11 &       0.0147 &              0.0 &                 0.0087 &                        1.0 \\
      12 &       0.0166 &              0.0 &                 0.0101 &                        1.0 \\
      13 &       0.0220 &              0.0 &                 0.0133 &                        1.0 \\
      14 &       0.0440 &              0.0 &                 0.0260 &                        1.0 \\
      15 &       0.0226 &              0.0 &                 0.0298 &                        1.0 \\
      16 &      22.0007 &              0.0 &                 4.0632 &                        1.0 \\
      17 &       0.4948 &              0.0 &                 0.1015 &                        1.0 \\
      18 &       0.7157 &              0.0 &                 0.0558 &                        1.0 \\
      19 &      -5.7637 &              0.0 &                 1.0304 &                        1.0 \\
      20 &       0.2148 &              0.0 &                 0.0758 &                        1.0 \\
      21 &       2.3658 &              0.0 &                 0.3694 &                        1.0 \\
      22 &       0.1997 &              0.0 &                 0.0816 &                        1.0 \\
\bottomrule
\end{tabular}

\end{center}

Same for test data, also sowing the before and after mean/standard deviation:
\begin{center}
\begin{tabular}{rrrrr}
\toprule
 Feature &  before mean &  normalized std &  before std &  normalized mean \\
\midrule
       1 &     152.4790 &          0.8557 &     37.9096 &          -0.0786 \\
       2 &     189.3091 &          0.8455 &     82.9900 &          -0.1580 \\
       3 &     117.6037 &          0.8931 &     40.8634 &           0.0556 \\
       4 &       0.0064 &          1.4108 &      0.0056 &           0.1132 \\
       5 &       0.0000 &          1.2908 &      0.0000 &           0.0716 \\
       6 &       0.0034 &          1.4618 &      0.0035 &           0.0869 \\
       7 &       0.0036 &          1.3865 &      0.0032 &           0.1157 \\
       8 &       0.0102 &          1.4621 &      0.0104 &           0.0870 \\
       9 &       0.0317 &          1.3311 &      0.0212 &           0.2490 \\
      10 &       0.3023 &          1.3524 &      0.2200 &           0.2452 \\
      11 &       0.0167 &          1.3105 &      0.0113 &           0.2296 \\
      12 &       0.0192 &          1.3334 &      0.0135 &           0.2509 \\
      13 &       0.0262 &          1.4799 &      0.0197 &           0.3166 \\
      14 &       0.0500 &          1.3105 &      0.0340 &           0.2296 \\
      15 &       0.0271 &          1.6319 &      0.0486 &           0.1491 \\
      16 &      21.7701 &          1.1666 &      4.7401 &          -0.0568 \\
      17 &       0.5023 &          1.0405 &      0.1057 &           0.0736 \\
      18 &       0.7205 &          0.9754 &      0.0544 &           0.0868 \\
      19 &      -5.6042 &          1.1030 &      1.1365 &           0.1548 \\
      20 &       0.2383 &          1.1674 &      0.0885 &           0.3107 \\
      21 &       2.3981 &          1.0647 &      0.3933 &           0.0874 \\
      22 &       0.2135 &          1.1894 &      0.0971 &           0.1686 \\
\bottomrule
\end{tabular}

\end{center}

So we can see now that each deviation in the training data end up to be 1 and the mean 0.
But since we also use the empirical mean and standardized deviation of the training data for the test data we won't end up exactly on 1 or 0.
But in the end the normalized means and normalized standard eviations end up still vey much more near to 1 or 0 than the original ones.

% model selection
\subsection{Model selection using grid-search}

My selection for the logarithmic scale for $y$ and $C$, by setting $C=10$, and $y=0.1$ in the middle of the scale:
\begin{align}
\mathcal{C}=\{0.01,0.1,1,10,100,1000,10000\}\\ 
\mathcal{Y} = \{0.0001,0.001,0.01,0.1,1,10,100 \}
\end{align}

I implemented the 5-cross validation with \texttt{GridSearchCV} from the python module \texttt{sklearn.model\_selection} in the \texttt{sklearn} library.
Where we calculate each cross validation score for all pairs of $(C,y)$, the pair with the highest score would be then the best hyperparamter pair configuration we are searching for.
So that I ended up using a heatmap to show all the cross validation scores from which we can read out the best possible configuration.

\newpage
\begin{figure}[!htpb]
\begin{tabular}{rrrrrrrr}
\toprule
  & 0.0001   &  0.0010   &  0.0100   &  0.1000   &  1.0000   &  10.0000  &  100.0000 \\
  &         &           &           &           &           &           &           \\
\midrule
 0.01 & 0.734694 &  0.734694 &  0.734694 &  0.734694 &  0.734694 &  0.734694 &  0.734694 \\
 0.1 & 0.734694 &  0.734694 &  0.734694 &  0.734694 &  0.734694 &  0.734694 &  0.734694 \\
 1.0 & 0.734694 &  0.734694 &  0.867347 &  0.897959 &  0.795918 &  0.734694 &  0.734694 \\
 10.0 & 0.734694 &  0.877551 &  0.897959 &  \textcolor{red}{0.908163} &  0.795918 &  0.775510 &  0.734694 \\
 100.0 & 0.877551 &  0.887755 &  0.867347 &  0.908163 &  0.795918 &  0.775510 &  0.734694 \\
 1000.0 & 0.887755 &  0.846939 &  0.877551 &  0.908163 &  0.795918 &  0.775510 &  0.734694 \\
 10000.0 & 0.846939 &  0.877551 &  0.877551 &  0.908163 &  0.795918 &  0.775510 &  0.734694 \\
\bottomrule
\end{tabular}
\caption{Showing all the cross validation scores in a table}
\end{figure}


Either from the table above or getting it from the outputs of my implementation, we will see that that the best validation score we get with the hyperparameters $\{C=10,y=0.1\}$.
Is the validation score of $0.908163\approx$.
Additionally we can calculate the accuracy of the cross validation which is: $0.907216\approx$

\subsection{Inspecting the kernel expansion}
For the purpose of calculating bounded and free bounded vectors I used \texttt{sklearn} again, by fitting the data again as in the exercise before and then keep $y$ the same while going through various values of $C$.
Then I came to following solutions:
\begin{minted}{text}
	C = 0.1,   bounded support vectors: 54, free support vectors: 0
	C = 10,    bounded support vectors: 23, free support vectors: 17
	C = 100,   bounded support vectors: 12, free support vectors: 20
	C = 1000,  bounded support vectors: 1 , free support vectors: 26
	C = 10000, bounded support vectors: 0 , free support vectors: 26 
\end{minted}
As we see above the value of $C$ affects the misclassification rate of training examples. For small values of $C$ a large-margin hyperplane will be still used regardless of the misclassification rate.
For larger values of C, the SVM will pick a smaller-margin hyperplane provided it minimizes the misclassification rate.

